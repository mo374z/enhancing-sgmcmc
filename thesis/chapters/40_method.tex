\section{Method}\label{sec:method}
\subsection{Background \& General Feasibility}\label{sec:general_idea}

\begin{itemize}
    \item best case would be to compute the inverse Fisher Information Matrix in each step as preconditioning to apply to local geometry of the current position
    \item cannot be done in our MCMC preconditioning context since it is much to expensive to estimate the FIM in every step from scratch
    \item we cannot do a Hessian approximation like RmsProp or Adam, since they utilize values depending which depend on previous steps. This does then not meet meet the markov property (MCMC theory breaks)
    \item this is why we optimize first obtaining a high likelihood soultion and a rough estimation of the local geometry (via Adam like Hessia approximation or IVON)
    \item using this information might help shorten the warmup/burnin phase without hurting the performance
    \item during sampling we then want to efficiently sample the a local subspace of the high dimensional posterior where the optimizer landed
    \item multimodality is tackled through ensembling and lengthy markov chains
\end{itemize}

\subsection{Metrics}
\begin{itemize}
    \item explain how impact of preconditioning is measured and what we aim for
\end{itemize}

\subsection{Approach}
\begin{itemize}
    \item formalize the preconditioning step
\end{itemize}

\newpage
