\chapter{Introduction}\label{chap:introduction}

\section{Motivation}

\begin{itemize}
    \item best case would be to compute the inverse Fisher Information Matrix in eqch stap as preconditioning to apply to local geometry of the current position
    \item cannot be done in our \acrshort{MCMC} preconditioning context since it is much to expensive to estimate the FIM in every step from scratch
    \item we cannot do a Hessian approximation like RmsProp or Adam, but this depens on previous steps and therefor does not meet the markov property (MCMC theory breaks)
    \item this is why we optimize first optaining a high likelihood soultion and a rough estimation of the local geometry (via Adam like Hessia approximation or IVON)
    \item using this information might help with shortening the warmup/burnin phase without hurting the performance
    \item during sampling we then want to efficiently sampe the local part of the high dimensional posterior
    \item multimodality is then represented through ensembling and lengthy markov chains
\end{itemize}


\section{Problem}

\begin{itemize}
    \item in high dimensional, complex posteriors we are often confronted with ill-conditioned Hessians
    \item this might lead to slow convergence when the sampler is not able to accord to this
    \item we can show this effect on some simple gaussian (mixture) examples
\end{itemize}

Note: the visualizations show the true density in the background, while compute the steps of the MCMC sampler on samples of the log density.


\section{Approach}

\begin{itemize}
    \item use preconditioning
\end{itemize}

% In many cases, the thesis will be about a new approach to a known problem.
% In this case, describe the approach in reasonably general terms, without getting too technical.
% The reader should understand what the new idea is that is being used.

% If the thesis is more about highlighting a new aspect of known methods, e.g.\ evaluating runtime complexity of methods for which runtime was not systematically measured in the literature, this should be described here instead (possibly with a title that is not ``Approach'').

% It may help to define various ``research questions'' that are being investigated.
% These can even be numbered and referred to in later parts of the text, e.g.\ to describe which experiments were conducted to answer which kinds of questions.

\section{Outline}

% Outline of the thesis.
% What is in each chapter? What is the structure of the thesis?
% E.g.\ ``\cref{chap:background} presents the theoretical background that is relevant for the thesis, particularly mathematical concepts and algorithms''.
